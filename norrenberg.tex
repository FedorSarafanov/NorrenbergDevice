\input{text/diss}
\usepackage{setspace}

\begin{document}

\def\labauthors{Понур К.А., Сарафанов Ф.Г., Сидоров Д.А.}
\def\labgroup{420}
\def\labnumber{000}
\def\labtheme{Изучение явлений двулучепреломления и поляризации света на приборе Норренберга}
\renewcommand{\vec}{\mathbf}
\renewcommand{\Re}{\operatorname{Re}}
\renewcommand{\Im}{\operatorname{Im}}
\renewcommand{\phi}{\varphi}
\renewcommand{\kappa}{\varkappa}
\renewcommand{\hat}{\widehat}
%%%%%%%%%%%%%%%%%%%%%%%%%%%%%%%%%%%%%%%%%%%%%%%%%%%%%%%%%%%%%%%%%%%%%%%%%%%%%%%
\input{text/titlepage}
%%%%%%%%%%%%%%%%%%%%%%%%%%%%%%%%%%%%%%%%%%%%%%%%%%%%%%%%%%%%%%%%%%%%%%%%%%%%%%%
\begin{spacing}{1}
\tableofcontents
\end{spacing}
% \setstretch{1.2}
\newpage
%%%%%%%%%%%%%%%%%%%%%%%%%%%%%%%%%%%%%%%%%%%%%%%%%%%%%%%%%%%%%%%%%%%%%%%%%%%%%%%
 
\section{Введение}
\subsection{}

\section{Экспериментальная часть}
\subsection{Часть 1}
\subsubsection{Исландский шпат}

Пронаблюдали двулучепреломление и поляризацию света в крестраже исландского шпата. 

Определили направление оптической оси:
\begin{figure}[H]
	\centering
	\includegraphics[width=\textwidth]{pic/rot_eo.jpg}
	\caption{Положение оптической оси}
	\label{fig:figure1}
\end{figure}

Нашли плоскость главного сечения, указали направление колебаний вектора $E$ o- и e- волн.
\begin{figure}[H]
	\centering
	\includegraphics[width=\textwidth]{pic/eo.jpg}
	\caption{Положение плоскости главного сечения: образуется векторами $\vec{k}$ и оптической осью. Направление колебаний вектора $E$ в обыкновенной и необыкновенной волнах}
	\label{fig:figure1}
\end{figure}

\subsubsection{Полярископ}

Пронаблюдали двулучепреломление естественного света. Для этого рассмотрели светящуюся точку на кристалле. Точка, отвечающая обыкновенной волне, неподвижна при небольшом повороте кристалла и если убрать кристалл, остается на прежнем месте, а отвечающая e-волне, смещается.

\begin{figure}[H]
	\centering
	\includegraphics[width=\textwidth]{pic/dv.jpg}
	\caption{Положение точек при вращении}
	\label{fig:figure1}
\end{figure}

Вращая кристалл, получили четыре относительных положения обеих точек и положений плоскости главного сечения (относительно друг друга остаются неподвижными)

Сделали вывод, что изображение необыкновенной точки <<ближе к глазу>>

Наблюдая за кристаллом через вращаемый анализатор, заметили чередование максимума яркости одной точки и минимума другой. Смена происходит через каждые $\pi\over2$, это обосновывается перпендикулярной поляризацией e- и o- волн.

%%%%%%%%%%%%%%%%%%%%%%%%%%%%%%%%%%%%%%%%%%%%%%%%%%%%%%%%%%%%%%%%%%%%%%%%%%%%%%%
\newpage
\section{Заключение}

\end{document}