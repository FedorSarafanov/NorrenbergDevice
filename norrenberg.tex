\input{text/diss}
% \usepackage{subfig}
\usepackage{setspace}

\begin{document}
\def\labauthors{Понур К.А., Сарафанов Ф.Г., Сидоров Д.А.}
\def\labgroup{420}
\def\labnumber{312}
\def\labtheme{Изучение явлений двулучепреломления и поляризации света на приборе Норренберга}
% \def\labtheme{Хугарлын болон туйлшралын үзэгдлийн судалгаа Норенберг жингийн гэрэл}
\renewcommand{\vec}{\mathbf}
\renewcommand{\Re}{\operatorname{Re}}
\renewcommand{\Im}{\operatorname{Im}}
\renewcommand{\phi}{\varphi}
\renewcommand{\kappa}{\varkappa}
\renewcommand{\hat}{\widehat}
%%%%%%%%%%%%%%%%%%%%%%%%%%%%%%%%%%%%%%%%%%%%%%%%%%%%%%%%%%%%%%%%%%%%%%%%%%%%%%%
\input{text/titlepage}
%%%%%%%%%%%%%%%%%%%%%%%%%%%%%%%%%%%%%%%%%%%%%%%%%%%%%%%%%%%%%%%%%%%%%%%%%%%%%%%
\begin{spacing}{1}
\tableofcontents
\end{spacing}
% \setstretch{1.2}
\newpage
%%%%%%%%%%%%%%%%%%%%%%%%%%%%%%%%%%%%%%%%%%%%%%%%%%%%%%%%%%%%%%%%%%%%%%%%%%%%%%%
 
\section*{Введение}
\addcontentsline{toc}{section}{Введение}
Прибор Норренберга - универсальный учебный поляризационный
прибор. Его можно использовать и как полярископ и как поляриметр,
Он дает возможность получать полярное ванный свет, изменить поляризацию света, анализировать поляризованный свет, наблюдать интерференцию в поляризованном свете и измерять углы поворота плоскости поляризации в оптически активных средах.

Прибор Норренберга состоит из поляризатора, анализатора,
трех поворотных столиков, револьверной обоймы для быстрой смены
светофильтра и специального осветителя, смонтированного на общем основании с прибором.

Поляризатором служит прозрачное стеклянное зеркало в оправе, допускающей поворот вокруг горизонтальной оси и установку зеркала под углом Брюстера к отраженному вверх или вниз пучку света. Как известно, свет отраженный под углом Брюстера пол­ностью линейно поляризован. Световая трубка осветителя имеет
лишь две необходимые степени свободы, а именно: перемещение ее вверх и вниз (по двум вертикальным направлениям) и поворот вокруг горизонтальной оси, этого достаточно для направления светового пучка на поляризующее зеркало под углом Брюстера.

Особенностью прибора Норренберга нашей конструкции являет­ся возможность быстрой установки его как по обычной, простой схеме, так и по схеме <<удвоителя Норренберга>>. В удвоителе поляризованный свет проходит через исследуемый образец дважды (вниз и после отражения от вспомогательного
горизонтального зеркала вверх, по направлению к глазу наблюдателя). Двойное прохождение света через образец позволяет наблюдать ряд новых интересных явлений.

Верхний столик предназначен для установки и поворота смен­ных анализаторов, которыми поочередно могут быть поляризацион­ная призма из исландского шпата, пленочный поляроид в специальной оправе, черное зеркало и стопа стеклянных пластин.

Исследуемые образцы обычно помещают на среднем столике, а в схеме удвоителя их можно помещать как на среднем, так и на нижнем столике прибора.

Настроенный прибор Норренберга посылает на исследуемый образец линейно поляризованный свет с колебаниями вектора $\vec{E}$, происходящими перпендикулярно плоскости падения в направлении
перпендикуляра, соединяющего стойки прибора.
% \section{Экспериментальная часть}
% \part{}
% \section{Теоретическая часть}
\subsection*{Двулучепреломление и поляризация в кристаллах}
\addcontentsline{toc}{subsection}{Двулучепреломление и поляризация в кристаллах}

Двойное преломление открыто в 1669 году датским ученым Зразмом Бартолином в кристаллах кальцита ($CaCO_3$), привезенных из Исландии. По месту находки кальцит назвали исландским шпа­том. В 1690 году голландский физик Христиан Гйюгенс открыл поляризацию света при двойном преломлении в кристаллах кальцита.

Природный кальцит легко колется по плоскостям спайности
параллельным граням ромбоэдра. Поэтому ему нетрудно придать
форму ромбоэрической призмы или ромбоэдра. Ромбоэдр - это шестигранник, все шесть граней которого одинаковые ромбы. Его можно
рассматривать как куб, сжатый вдоль одной из его пространствен­ных диагоналей. Каждая из восьми вершин ромбоэдра есть вершина
трехгранного угла, но две из них резко отличны от шести остальных. Две особенные вершины образованы тремя одинаковыми плоскими углами, равными 102$^\circ$. Каждая из шести остальных вершин образована  одним тупым и двумя острыми углами в 78$^\circ$.

Прямая, соединяющая две особенные вершины, есть короткая
пространственная диагональ ромбоэдра и является его геометрической осью симметрии третьего порядка. При вращении ромбоэдра
вокруг этой оси он через каждую треть полного оборота приходит
в самосовпадение, занимая положение тождественное с исходным.

Направление оптической оси в кристалле $CaCO_3$ параллельно короткой пространственной диагонали ромбоэдра и образует углы в $63^\circ45'$ и $45^\circ22'$ с его ребрами и гранями, соответственно.
Плоскость, проходящую через падающий луч и направление оптической оси в точке падения, называют главной плоскостью или
главным сечением двоякопреломляющего кристалла.

Бездефектные кристаллы кальцита до сих пор являются наилучшим материалом для изготовления высококачественных поляризующих
свет призм - поляризаторов. Первая поляризующая призма из кальцита была изготовлена в 1828 году шотландским физиком Уильямом
Николем. По имени изобретателя ее назвали: <<призма Николя>> или
кротко <<николь>>. 

Из-за присущих им недостатков причин Николя уже
давно не изготовляют, но термин <<николь>> глубоко укоренился в
учебной и научной литературе и употребляется как синоним любого
типа поляризатора или анализатора. Словом <<николь>> называют и
поляризаторы и анализаторы независимо от их устройства и принципа действия.

Для изготовления призмы Николя сильно удлиненную ромбоэдрическую призму из кальцита разрезают по плоскости, перпендикулярной главному сечению, на две равные части и склеивают их вновь канадским бальзамом.

Обыкновенный луч, с колебаниями перпендикулярными главному сечению,
испытывает полное отражение на границе с канадским бальзамом и
поглощается на зачерненных боковых гранях призмы. Необыкновенный
луч, с колебаниями в плоскости главного сечения, проходит через
обе половинки призмы и дает линейно поляризованный свет на ее
выходе.

Практически все современные поляризационные призмы из каль­цита, подобно призме Николя, состоят из двух склеенных половинок,
но отличаются от нее в лучшую сторону тем, что плоскости их входного и выходного сечений перпендикулярны продольной оси призмы
и оси выходящего из нее поляризованного пучка.

Опыты Гюйгенса по наблюдению двойного преломления и поляризации света в кристаллах кальцита выполняются в нашей лаборатории на специальном полярископе с точечным источником
света. Точечным источником служит светодиод АЛ 1028, питаемый
от батарейки из двух элементов типа <<373>>. Кристалл кальцита
ставится на поворотный столик с центральным отверстием, под ко­
торым находится выдвижной пленочный поляроид в оправе и свето­диод.

Для наблюдения двойного преломления в естественном (не поляризованном) свете светящуюся точку рассматривают сквозь
кристалл, через центрированный наглазник и поляроидный анализатор, располагаемые поочередно в специальном гнезде,
на расстоянии наилучшего зрения от кристалла.

Для наблюдения в поляризованном свете в пространство меж­ду светодиодом и кристаллом вдвигается пленочный поляроид -- поляризатор, а наглазник, в случае необходимости, заменяется поляроидным анализатором в идентичной с наглазником оправе.


\subsection*{Поляризация световых волн при отражении и преломлении}
\addcontentsline{toc}{subsection}{Поляризация световых волн при отражении и преломлении}

Энергитический коэффициент отражения плоской световой волны по определению равен
\begin{equation}
	\rho=\frac{P_1}{P}=\frac{\bar{E_1}^2}{\bar{E}^2}
\end{equation}
Здесь $P_1$ и $P$, $\bar{E_1}^2$ и $\bar{E}^2$ - соответственно потоки энергии и средние по времени квадраты напряженности электрического поля в отраженной и падающей волне.

Из формулы Френеля следует, что коэффициент отражения $\rho$ от плоской границы двух диэлектриков зависит от направления вектора $\vec{E}$ в падающей волне. Так, если вектор $\vec{E}$ перпендикулярен к плоскости падения, то 
\begin{equation}
\label{eq:2.1}
	\rho_{\bot}=\frac{\sin^2(\phi-\psi)}{\sin^2(\phi+\psi)},
\end{equation}
где $\phi$- угол падения, а $\psi$- угол преломления.

Если же вектор $\vec{E}$ параллелен плоскости падения, то
\begin{equation}
\label{eq:2.2}
	\rho_{\|}=\frac{\tg^2(\phi-\psi)}{\tg^2(\phi-\psi)}
\end{equation}

Из (\ref{eq:2.1}) и (\ref{eq:2.2}) видно, что:

1. При любом $\phi$ $\rho_{\bot}>\rho_{\|}$

2. При $\phi+\psi=\frac{\pi}{2} \label{eq:2.3} \, (\ref{eq:2.3})$
$\rho_{\bot}\neq0$ при $\rho_{\|}$.

Это означает, что в  отраженном свете всегда преобладает $E_{\bot}$.
При отражении под углом Брюстера, удовлетворяющим условию (\ref{eq:2.3})
отраженный свет полностью линейно поляризован. При этом преломленный свет оказывается частично поляризованным.

Угол отражения, удовлетворяющий соотношению (\ref{eq:2.3}), называют углом поной поляризации или углом  $\phi_{\text{Б}}$ по имени шот­ландского ученого Давида Брюстера, открывшего в 1812 году полную
поляризацию при отражении. Из закона преломления
\begin{equation}
	\frac{\sin{\phi}}{\sin{\psi}}=n
\end{equation}
и условия (\ref{eq:2.3}) следует:
\begin{equation}
\label{eq:ba}
	\tg\phi_{\text{Б}}=n
\end{equation}
В качестве числовой характеристики частично поляризованного света
используется понятие степени поляризации. Так, степень поляризации
отраженного света по определению равна
\begin{equation}
 	\Delta=\frac{I_{\bot}-I_{\|}}{I_{\bot}+I_{\|}}
% 	\frac{\bar{E_{\bot}^2-\bar{E^2_{\|}}{\bar{E_{\bot}^2+\bar{E^2_{\|}}}
\end{equation}
Аналогично определяется степень поляризации преломленного света
и света, прошедшего через прозрачную плоскопараллельнуго пластинку.

Прохождение света через пластинку эквивалентно двум последо­
вательным преломлениям на ее первой и второй границе с воздухом.
При каждом преломлении степень поляризации возрастает но несколь­ко процентов (порядка $8\%$ при n= 1.52). Пропустив свет через
стопу из достаточно большого числа ( $N\approx20$) тонких пластин,
можно получить степень поляризации близкую к единице.

Таким образом, прозрачное стеклянное зеркало, зеркало из
черного стекла и стопу стеклянных пластин можно использовать в
качестве поляризатора или анализатора.

\subsection*{Фазовые пластинки, вращение поляризации}
\addcontentsline{toc}{subsection}{Фазовые пластинки, вращение поляризации}

% \section{Интерференция поляризованного света в параллельных лучах.
% Фазовые пластинки $\lambda/2$ и $\lambda/4$ и образцы для исследования. Вращение плоскости колебаний в кристаллах кварца. Опыты с удвоителем Норренберга}

% \begin{figure}[H]
%     \centering
%     \subfloat[label 1]{{\includegraphics[width=5cm]{ris/ris1.pdf} }}%
%     \qquad
%     \subfloat[label 2]{{\includegraphics[width=5cm]{ris/ris1b.pdf} }}%
%     \caption{2 Figures side by side}%
%     % \label{fig:example}%
% 	\label{fig:figure5}
% \end{figure}
\begin{figure*}[t!]
    \centering
    \begin{subfigure}[c]{0.5\textwidth}
        \centering
        \includegraphics[height=8cm]{ris/ris1.pdf}
        \caption{Установка}
    \end{subfigure}%
    ~ 
    \begin{subfigure}[c]{0.5\textwidth}
        \centering
        \includegraphics[height=5cm]{ris/ris1b}
        \caption{Ориентация поляризатора и анализатора}
    \end{subfigure}
    \caption{Иллюстрации к объяснению}
% 	\label{fig:figure5}

\end{figure*} 

Пусть линейно поляризованный свет, прошедший через узкопо­лосный фильтр $\text{Ф}$ и поляризатор $\text{П}$ , падает на двупреломляющую пластинку толщины d нормально к ее большим граням в
положительном направлении оси $Z$ (рис.\ref{fig:figure5} ). Пусть, кроме того,
большие грани пластинки параллельны оптической оси $X$ , образующей углы $\alpha$ и $\beta$ с направлениями колебаний вектора $\vec{E}$
на выходе поляризатора П анализатора А соответственно
(рис. \ref{fig:figure5} ). Электрическое поле $\vec{E}$ в падающем на пластинку параллель­ном пучке можно приближенно описать уравнением плоской бегущей
волны 
\begin{equation}
	\vec{E}(t)=\vec{E}\cos(\omega t-kz)
\end{equation}



На входной грани $Z=О$ поле $\vec{E}\cos(\omega t)$ возбуждает
в пластинке два взаимно перпендикулярных колебания $\vec{E_y}$
(обыкновенная волна) и $\vec{E_x}$ (необыкновенная волна) с амплитудами $E\sin\alpha$ и $E\cos\alpha$ и разными скоростями
распространения вдоль оси Z
\begin{gather}
	V_z^{\text{об}}=V_o=\frac{c}{n_0} \\
	V_z^{\text{необ}}=V_e=\frac{c}{n_e}
\end{gather}

Здесь $n_0$ и $n_e$ -- главные показатели преломнелия обыкновенной и необыкновенной волны.

На выходе из пластинки получаются взаимоперпендикулярные колебания
\begin{gather}
	E_y=E\sin\alpha\cos(\omega t -k_yd) \\
	E_z=E\cos\alpha\cos(\omega t -k_zd),
\end{gather}
где
\begin{gather}
	k_y=\frac{\omega}{V_0}=\frac{\omega}{c}\cdot n_0=kn_0, \,
	k_x=\frac{\omega}{V_e}=\frac{\omega}{c}\cdot n_e=kn_e
\end{gather}
Эти колебания сдвинуты по фазе на величину
\begin{equation}
	\label{eq:3.1}
	\delta=(k_x-k_y)d=k(n_e-n_0)d=\frac{2\pi}{\lambda}\Delta nd,
\end{equation}
зависящую от длины волны $\lambda$ и свойств пластинки.

Разность главных показателей преломления $\Delta n=n_e-n_0$
называют величиной двупреломления или просто <<двупреломлением>>.
На вход анализатора А поступает суперпозиция двух векторных
взаимно перпендикулярных колебаний с неравными амплитудами
$E\sin\alpha$ и $E\cos\alpha$ и постоянным сдвигом фаз $\delta$, т.е. эллиптически поляризованное колебание, которое
в специальных случаях может переходить в поляризованное по кру­гу 
(при $\alpha=\pm\frac{\pi}{4}$ и $\delta=\pm\frac{\pi}{2}$. При эллиптической
поляризации конец вектора Е движется (в фиксированной плоскости
XY)ио эллипсу, ориентация которого зависит от амплитуд
$E\sin\alpha$, $E\cos\alpha$ и разности фаз $\delta$ . Из-за
зависимости $\delta(\lambda)$ положение эллипса поляризации изменяется
с изменением длины волны $\lambda$.

Анализатор А, расположенный на расстоянии $Z-d$ от
пластинки, пропускает а глаз наблюдателя сумму двух параллельных
колебаний 
\begin{gather}
	E=E\cos\alpha\cos\beta\cos\left[\omega t -k_xd-k(z-d) \right]+\\
	+E\sin\alpha\sin\beta\cos\left[\omega t -k_yd-k(z-d) \right]
\end{gather}
с приобретенным в пластинке сдвигом фаз $\delta$. Тем самым создает­
ся возможность наблюдать интерференцию в параллельном поляризо­
ванном свете. Для описания интерференционной картины необходимо
знать интенсивность света I на выходе анализатора, которая с
точностью до постоянного размерного множителя равна квадрату ам­
плитуды Е ($I\approx E^2$). Чтобы найти и получить для I выражение
удобное для интерпретации, - обозначим $E\cos\alpha\cos\beta=a_1$,
$E\sin\alpha\sin\beta=a_2$  и воспользуемся известной формулой
для квадрата амплитуды суммарного колебания:
\begin{equation}
	E^2=a_1^2+a_2^2+2a_1a_2\cos\delta=(a_1+a_2)^2-4a_1a_2\sin^2\frac{\delta}{2}
\end{equation}
После несложных тригонометрических преобразований, получаем
\begin{equation}
	\label{eq:1}
	I=I\left[\cos^2(\beta-\alpha)-\sin2\alpha\sin2\beta\sin^2\frac{\delta}{2} \right]
\end{equation}
Формула (\ref{eq:1}) описывает все возможные случаи прохождения параллельных линейно поляризованных лучей через пластинку, параллельную
оптической оси, и анализатор. Особый интерес представляет слу­чай наблюдения в скрещенных и в параллельных николях.

При скрещенных николях $\beta-\alpha=\frac{\pi}{2}$ и
\begin{equation}
	\label{eq:3bot}
	I_{\bot}=I\sin^22\alpha \sin^2\frac{\delta}{2}
\end{equation}

При параллельных николях $\beta-\alpha=0$ и 
\begin{equation}
	\label{eq:3par}
	I_{\|}=I\left(1-\sin^22\alpha \sin^2\frac{\delta}{2}\right)
\end{equation}

Следовательно
\begin{equation}
\label{eq:2}
I_{\bot}+I_{\|}=I	
\end{equation}
Соотношения (\ref{eq:2}) справедливы для каждой спектральной компоненты в случае падающего на устройство белого света. Суммируя равенства (\ref{eq:2})
для компонент белого света получим
\begin{equation}
	\sum_{\lambda} I_{\bot}+\sum_{\lambda} I_{\|}=\sum_{\lambda} I
\end{equation}
Как видно из формулы (\ref{eq:1}) интенсивность в интерференционной
картине зависит от величины $\delta$. Рассматривая двупреломляющую
пластинку через анализатор, глаз будет видеть ее светлой, темной
или в разных местах различно окрашенной в зависимости от эначе-
ния и распределения сдвига фаз $\delta$ по поверхности пластины.
Сам сдвиг фаз $\delta$ , как видно из (\ref{eq:3.1}), зависит от порядка интерференции
 $m=\frac{\Delta nd}{\lambda}$. При низких порядках интерферен­ции наблюдаются цветные интерференционные окраски, появление ко­торых следует из структуры формулы (\ref{eq:1}), в которой первое слага­емое $\cos^2(\beta-\alpha)$ постоянно для всех цветов, а второе -
содержащее множитель $\sin^2\frac{\delta}{2}$, зависит от $\lambda$. Наиболее сочные интерференционные окраски наблюдаются в скрещенных
николях, когда $\cos^2(\beta-\alpha)=0$ , а амплитуды $E\sin\alpha$
и $E\cos\alpha$ одинаковы. Положение пластинки, при котором
$E\cos\alpha=E\sin\alpha$ и $\alpha=\pm\frac{\pi}{4}$ называется <<диаго­
нальным>>, оно особенно благоприятно для наблюдения интерференци­
онных окрасок при освещении белым светом.

Для изменения характера поляризации и анализа поляризован­
ного света применяют фазовые пластинки в половину и в четверть
длины волны. Это двупреломляющие пластинки, параллельные опти­
ческой оси, на выходе из которых сдвиг фаз между обыкновенной и
необыкновенной волной равен $\pm\pi$ или $\pm\frac{pi}{2}$, соответствен­
но. Толщины полуволновых и четверть-волновых пластинок вычисляются из условия $\Delta d=\frac{\lambda}{2}$ и $\Delta nd=\frac{\lambda}{4}$.Фазовые
пластинки чаще всего изготовляют из таких природных кристаллов,
как исландский шпат ($CaCO_3$), кварц ($SiO_2$) и слюда.

При работе с образцами из кварца следует иметь в виду, что
кварц, наряду с двойным преломлением, обладает еще одним удивительным свойством совсем иной природы, а именно - способностью
вращать плоскость колебаний линейно поляризованного света. Эта
способность называется оптической активностью и обусловлена особым, винтовым расположением молекул $SiO_2$ в кристаллической решетке кварца. Вращение плоскости колебаний наиболее сильно
выражено именно тогда, когда двойное преломление отсутствует, т.
е. при распространении света в кристалле по направлению оптичес­
кой оси.

В природе встречаются две модификации кристаллов кварца -
правей и левая. В правых и левых кристаллах плоскости колебаний
поворачиваются в противоположных направлениях. Если для наблюдателя, смотрящего навстречу световому лучу, плоскость колебаний
на выходе из кварца повернута по часовой стрелке, то такой кварц
называется правовращающим или правым. 

Для того, чтобы избежать
неоднозначности при экспериментальном определении направления
вращения в кварце, толщина исследуемого образца должна быть достаточно мала (не более одного миллиметра).

Угол поворота плоскости колебаний в кварце пропорционален
толщине образца d и равен $\alpha=\alpha_1d$, где $\alpha_1$-удельное вращение, измеряемое в град/мм. Величина $\alpha_1$ сильно зависит от длины волны и быстро увеличивается при переходе от красного света к фиолетовому. 

Зависимость $\alpha_1(\lambda)$ называет­ся дисперсией вращения плоскости колебаний или дисперсией оптической активности.

Если на вход кварцевой пластинки поступает линейно поляризованный белый свет,то на ее выходе плоскости колебаний различных спектральных компонент поля $\vec{E}(\lambda)$ развернуты в <<цветовой
веер>>, от красного цвета к фиолетовому.

\subsubsection*{Бикварц}
\addcontentsline{toc}{subsubsection}{Бикварц}

Бикварц состоит из двух одинаковых полукруглых пластинок
из левого и правого кварца, склеенных по диаметральному сечению.
Толщина обеих половинок одинакова и равна 3,75 мм. 

При такой
толщине плоскости колебаний фиолетового света на выходе из би­кварца повернуты в противоположных направлениях на 180$^\circ$, а плос­кости колебаний наиболее ярких в спектре зеленовато-жеелтых лучей - на 90$^\circ$. Поэтому, в параллельных николях зеленовато-желтый
свет будет полностью погашен, а фиолетовый - полностью пропущен,
и при точной установке николей обе половинки бикварца будут окрашены в характерный серовато-фиолетовый оттенок.

При скрещенных николях окраска сменится на дополнительную
зеленовато-желтого оттенка. Малейшее отклонение одного из николей от параллельной или скрещенной установки вызывает резкое раз­личие в окраске обеих половинок бикварца. Поэтому бикварц позво­ляет установить николи в параллельное или скрещенное положение с более высокой точностью, чем при обычной установке их на темноту.

\newpage
\section{Часть 1 эксперимента}
\subsection{Исландский шпат}
\begin{figure}[H]
	\centering
	\includegraphics[width=0.65\textwidth]{pic/pic5.pdf}
	\caption{Кристалл исландского шпата, плоскость главного сечения $ABCD$}
	\label{fig:figure1}
\end{figure}
Пронаблюдали двулучепреломление и поляризацию света в кристалле исландского шпата. 

Определили направление оптической оси:
\begin{figure}[H]
	\centering
	\includegraphics[width=0.65\textwidth]{pic/rot_eo.jpg}
	\caption{Положение оптической оси}
	\label{fig:figure1}
\end{figure}

Нашли плоскость главного сечения, указали направление колебаний вектора $E$ o- и e- волн.
\begin{figure}[H]
	\centering
	\includegraphics[width=0.65\textwidth]{pic/eo.jpg}
	\caption{Положение плоскости главного сечения: образуется векторами $\vec{k}$ и оптической осью. Направление колебаний вектора $E$ в обыкновенной и необыкновенной волнах}
	\label{fig:figure1}
\end{figure}

\subsection{Полярископ}
\subsubsection{Двулучепреломление естественного света}

Пронаблюдали двулучепреломление естественного света. Для этого рассмотрели светящуюся точку на кристалле. Точка, отвечающая обыкновенной волне, неподвижна при небольшом повороте кристалла и если убрать кристалл, остается на прежнем месте, а отвечающая e-волне, смещается. Из наблюдения сделали вывод, что изображение необыкновенной точки <<ближе к глазу>>.

\begin{figure}[H]
	\centering
	\includegraphics[width=0.65\textwidth]{pic/dv.jpg}
	\caption{Положение точек при вращении}
	\label{fig:figure2}
\end{figure}

Вращая кристалл, получили четыре относительных положения обеих точек и положений плоскости главного сечения (относительно друг друга остаются неподвижными)


Наблюдая за кристаллом через вращаемый анализатор, заметили чередование максимума яркости одной точки и минимума другой. Смена происходит через каждые $\pi\over2$.

При этом яркости двух точек сравниваются при повороте на $\pi\over4$.

% , это обосновывается перпендикулярной поляризацией e- и o- волн.

\subsubsection{Двулучепреломление поляризованного света}

Определили направление вектора $\vec{E}$ после поляризатора. Для этого вращали кристалл, добиваясь исчезновения одной из точек. Это соответствует тому, что вектор колеблется либо в плоскости главного сечения и дает только необыкновенную волну, либо наоборот (перпендикулярно, обыкновенную). 

Сдвигая кристалл так, чтобы увидеть источник напрямую на краю кристалла, смотрели на точку: если она дергалась, то это необыкновенная волна (идет под углом, а когда убираем кристалл идет прямо -- наблюдаем скачок).

В нашем эксперименте первый же поворот дал точку o-волны, значит, мы можем определить геометрически из положения кристалла направление пропускания поляризатора: перпендикулярно текущему положению плоскости главного сечения.

Вращая кристалл, наблюдали чередование яркостей обыкновенной и необыкновенной точек. При угле между плоскостью пропускания поляризатора и плоскостью главного сечения $\pi\over2$ необыкновенная точка пропадала, при угле 0 и $\pi$ появлялась, но при этом пропадала обыкновенная точка. При угле $\pi\over4$ яркости точек сравнивались.


%%%%%%%%%%%%%%%%%%%%%%%%%%%%%%%%%%%%%%%%%%%%%%%%%%%%%%%%%%%%%%%%%%%%%%

\section{Часть 2 эксперимента}
\subsection{Черное зеркало}

По формуле (\ref{eq:ba}) при $n\approx1.5$ вычислили угол Брюстера $\phi_{\text{Б}}=56.5^{\circ}$. 

Заменили стандартный анализатор черным зеркалом и установили его под углом, близким к предполагаемому углу Брюстера. 

Нашли изображение в окне осветителя, повернули так, чтобы достигнуть максимального затемнения по вертикальной оси и подстроили точно зеркало поворотом вокруг горизонтальной оси, при этом угол составил 57 градусов.

Запротоколировали относительную ориентировку поляризующего и черного зеркал в максимальном и минимальном отражении от черного зеркала (90 градусов, сфотографировали оба положения):

\begin{figure*}[h!]
    \centering
    \begin{subfigure}[c]{0.5\textwidth}
        \centering
        \includegraphics[width=5cm]{pic/b1.jpg}
		\caption{Изображение окна осветителя}
		\label{fig:figure3}
    \end{subfigure}%
    ~ 
    \begin{subfigure}[c]{0.5\textwidth}
        \centering
        \includegraphics[width=5cm]{pic/b2.jpg}
		\caption{Изображение окна осветителя померкло}
		\label{fig:figure4}
    \end{subfigure}
    \caption{Относительная ориентировка}
% 	\label{fig:figure5}

\end{figure*} 
\subsection{Стопа стеклянных пластин}

Проведя ровно те же действия со стопой стеклянных пластин в проходящем свете, получили такие же результаты, с той разницей, что там где был максимум у черного зеркала, стал минимум у стопы, и наоборот.

\begin{figure}[H]
    \centering
    \begin{subfigure}[c]{0.5\textwidth}
        \centering
        \includegraphics[width=5cm]{pic/s90.jpg}
		\caption{Изображение окна осветителя}
		\label{fig:figure11}
    \end{subfigure}%
    ~ 
    \begin{subfigure}[c]{0.5\textwidth}
        \centering
        \includegraphics[width=5cm]{pic/s0.jpg}
		\caption{Изображение окна осветителя померкло}
		\label{fig:figure12}
    \end{subfigure}
    \caption{Относительная ориентировка}
% 	\label{fig:figure5}

\end{figure} 

% \begin{figure}[H]
% 	\begin{minipage}[h]{0.49\linewidth}
% 	\centering
% 	\includegraphics[width=0.5\linewidth]{pic/b2.jpg}
% 	\caption{Изображение окна осветителя померкло}
% 	\label{fig:figure4}
% 	\end{minipage}
% 	\hfill
% 	\begin{minipage}[h]{0.49\linewidth}
% 	\centering
% 	\includegraphics[width=0.5\linewidth]{pic/b1.jpg}
% 	\caption{Изображение окна осветителя}
% 	\label{fig:figure4}
% 	\end{minipage}
% \end{figure}
%%%%%%%%%%%%%%%%%%%%%%%%%%%%%%%%%%%%%%%%%%%%%%%%%%%%%%%%%%%%%%%%%%%%%%

\section{Часть 3 эксперимента}
\subsection{Определение типа пластинок}
Для определения анализатор ставили на затемнение(скрещенные поля, $\cos(\beta-\alpha)=0$).Пластинки клали между поляризатором и анализатором и вращали, следя за изменением интенсивности. Если интенсивность:
\begin{itemize}
	\item не изменялась, то пластинка является не фазовой.
	\item при изменении интенсивность, делали вывод о том, что пластинка фазовая, и необходимо определить является ли она "$ \lambda/2 $" или "$ \lambda/4 $".
\end{itemize}
\subsubsection{Фазовые пластинки}
Наблюдали за изменением интенсивности света в фазовых пластинках. Для этого выбрали белый свет, анализатор поставили на затемнение, чтобы $\cos(\beta-\alpha)=0$. Тогда
\begin{equation}
	I=I_0\sin2\alpha\sin2\beta\sin^2\frac{\delta}{2}
	\label{eq:2}
\end{equation}
Были определены две четверть-волновые, одна полувоенная и одна кварцевая пластинки. 
\subsubsection{Наблюдение в белом свете}


Убедились в том, что гашение поля зрения наблюдается 4 раза за оборот, через каждые $\pi/4$ со всеми образцами, параллельными оптической оси вне зависимости от их толщины.



\subsubsection{Наблюдения с узкополосным фильтром}
Согласно формуле (\ref{eq:1}) интенсивность не будет изменяться при вращении анализатора при установлении четверть-волновой пластинки в диагональное положение. 

Опытным путём мы определили из имеющихся пластинок пластинку $\frac{\lambda}{4}$ и получили с её помощью свет с круговой поляризацией.

Рассмотрим случай скрещенных николей и диагонально установленной полуволновой пластинки. В этом случае интенсивность на выходе из анализатора будет вдвое меньше интенсивности света, прошедшего через поляризатор (см.\ref{eq:3bot}). Также вне зависимости от поляризации падающего света, мы получим на выходе линейно поляризованный свет.

В случае параллельных николей (см.\ref{eq:3par}) интенсивность света на выходе анализатора будет вдвоем меньше интенсивности падающего света. Поляризация также будет полностью или частично линейной в зависимости от качества поляризатора. 

\subsubsection{Изменение цвета фазовых пластинок}

На белом (оранжевом) свете, вращая анализатор, наблюдали изменение окраски света с  фазовой пластинки: при 0 градусов -- оранжевый, с 60 до 120 скорее желтый чем оранжевый, при 180 -- опять оранжевый.

\section{Часть 4 эксперимента}
\subsection{Вращение поляризации в кристаллах кварца}
Определили угол и направление вращения плоскости поляризации в кварцевой пластинке, толщина которой 6 мм. Для этого выбрали определенный светофильтр, анализатор поставили на затемнение, положили пластинку на столик, при этом через анализатор опять стало видно свет. Повернули анализатор по и против часовой стрелки до затемнения. 
Поворот анализатора до и после установки пластинки  по часовой стрелке составил $84^{\circ}$ и $68^{\circ}$ соответственно для красного и зеленого, и против часовой стрелки $92^{\circ}$ и  $114^{\circ}$. 

\begin{table}[H]
	\caption{Определение угла поворота для кварцевой пластинки}
	\centering
	\begin{tabular}{|c|c|c|}
		\hline
		\multicolumn{ 1}{|c|}{светофильтр} & \multicolumn{ 2}{c|}{угол относительно часовой стрелки, град} \\ \cline{ 2- 3}
		\multicolumn{ 1}{|c|}{} & \multicolumn{1}{c|}{по} & против \\ \hline
		красный & 84 & \multicolumn{1}{c|}{92} \\ \hline
		зеленый & 68 & \multicolumn{1}{c|}{114} \\ \hline
	\end{tabular}
	\label{}
\end{table}

Если считать, что поляризация повернулась на острый угол, это соответствует углу поворота поляризации (правого вращения) 84 градуса для красного и 68 для зеленого.


Пронаблюдали в белом цвете чередование цветов в кварцевой пластинке:

\begin{figure}[H]
	\centering
	\includegraphics[width=10cm]{images/RADUGA.jpg}
	\caption{}
	\label{fig:figure6}
\end{figure}

\subsubsection{Бикварц}
Убедились в том, что половинки бикварца поворачивают плоскость поляризации в противоположных направлениях. Для этого выбрали красный светофильтр, положили бикварц на средний столик и, вращая анализатор, наблюдали, как меняется интенсивность в половинках бикварца. При этом наблюдалось затемнение то в одной половинке, то в другой, так как из предыдущих опытов следовало, что поляризация красного света поворачивается ровно на 90$ ^\circ $ в предположении, что толщина кварцевой пластинки и бикварца одинаковы.
\begin{figure}[h!]
	\caption{Интенсивность света в бикварце}
	\centering
	\begin{minipage}{0.3\linewidth}
		\includegraphics[width=\linewidth]{biquarz/red-l.jpg}
	\end{minipage}
	\begin{minipage}{0.3\linewidth}
		\includegraphics[width=\linewidth]{biquarz/red2.jpg}
	\end{minipage}
	\begin{minipage}{0.3\linewidth}
		\includegraphics[width=\linewidth]{biquarz/red-r.jpg}
	\end{minipage}
\end{figure}
% Наблюдали дополнительные цвета в половинках бикварца. Для этого выбрали белый свет. Вращая анализатор, выяснили, что дополнительные цвета совпадают для желтого цвета. Таким образом теоретически дополнительными цветами окажутся:

\subsection{Удвоитель Норренберга}
Установили прибор по схеме удвоителя Норренберга.


Убедились в отсутствии поворота плоскости поляризации через кварцевую пластинку. Для этого выбрали определенный светофильтр, анализатор поставили на затемнение, положили пластинку на нижний столик. При этом наблюдалось полное затемнение, что говорит об отсутствии поворота поляризации.

Наблюдали дополнительные цвета в половинках бикварца. Для этого выбрали белый свет. Вращая анализатор, выяснили, что дополнительные цвета совпадают для желтого цвета. 
% Норренберг хоёр дахин нэмэгдүүлсэн схемийн дагуу төхөөрөмжийг суулгасан.


% Бид кварцын тавцангаар туйлшруулагч онгоц эргэлдэж байгаагүй болохыг тогтоожээ. Үүнийг хийхийн тулд бид тодорхой гэрлийн шүүлтүүрийг сонгож, анализаторыг далдлах ажил хийсэн бөгөөд хавтанг доод хүснэгтэнд байрлуулсан. Энэ тохиолдолд туйлшралыг бүрэн зогсоогоод байгааг бүрэн харуулсан байна.

% Biquark-ын тал дахь нэмэлт өнгө ажиглав. Үүний тулд цагаан гэрэл сонгогдов. Анализаторыг эргүүлэхэд нэмэлт өнгө нь шар өнгөтэй таарч байгааг олж мэдсэн.
\begin{figure}[h!]
	\caption{Интенсивность света в бикварце}
	\centering
	\begin{minipage}{0.25\linewidth}
		\includegraphics[width=\linewidth]{biquarz/yellow-yellow.jpg}
	\end{minipage}
	\begin{minipage}{0.25\linewidth}
		\includegraphics[width=\linewidth]{biquarz/orange-green.jpg}
	\end{minipage}
	\begin{minipage}{0.25\linewidth}
		\includegraphics[width=\linewidth]{biquarz/red-blue.jpg}
	\end{minipage}
	\begin{minipage}{0.25\linewidth}
		\includegraphics[width=\linewidth]{biquarz/violet-violet.jpg}
	\end{minipage}
	\begin{minipage}{0.25\linewidth}
		\includegraphics[width=\linewidth]{biquarz/blue-red.jpg}
	\end{minipage}
	\begin{minipage}{0.25\linewidth}
		\includegraphics[width=\linewidth]{biquarz/green-orange.jpg}
	\end{minipage}
\end{figure}

%%%%%%%%%%%%%%%%%%%%%%%%%%%%%%%%%%%%%%%%%%%%%%%%%%%%%%%%%%%%%%%%%%%%%%%%%%%%%
\newpage
\section{Заключение}
% Мы, ничтоже сумнявшися, зело нагрешили да скатали сей свиток почти наполовину. Сопогами нашими могучими пропинали мы всея лабы и ничего не уразумели, но зато изучили поляризацию света, явления двойного лучепреломления и оптической активности в кристаллах при помощи универсального прибора Норренберга.
% Ми, не сумніваючись, зело нагрішили так скачали цей сувій майже наполовину. Сопогамі нашими могутніми пропіналі ми всієї Лаби і нічого не зрозуміли, але зате вивчили поляризацію світла, явища подвійного променезаломлення і оптичної активності в кристалах за допомогою універсального приладу Норренберга.
% Бид эргэлзээгүйгээр халуухан халааж, энэ гүйлгээ бараг талыг нь эргүүлэв. Бид хүчирхэг гараа хөдөлгөхийн тулд чадах бүхнээ ашиглаж чадсан ч бид ямар ч зүйлийг ойлгохгүй байсан ч бид туйлын туйлшралын судалгаа, нийтээрээ Норенбергерын тусламжтайгаар талстын оптик үйл ажиллагаа, оптик үйл ажиллагааны үзэгдэл зэргийг судлав.
% Слава Украiнi!!!
\end{document}