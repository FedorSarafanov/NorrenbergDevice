\documentclass[tikz]{standalone}

\usepackage[english,russian]{babel}
\usepackage[T2A,T1]{fontenc}
\usepackage[utf8]{inputenc}

\usepackage
{
    tikz,
    pgfplots,
    verbatim,
    amssymb,
}
\usepackage{xcolor}
\usepackage[europeanresistors,americaninductors]{circuitikz}
\usetikzlibrary
{
    arrows,
    patterns,
    angles,
    quotes,
    calc, 
    3d,
    backgrounds, 
    positioning
}

\makeatletter
\pgfkeys{/eye/.cd,
  x/.code           = {\def\eye@x{#1}},
  y/.code           = {\def\eye@y{#1}},
  rotation/.code    = {\def\eye@rot{#1}},
  radius/.code      = {\def\eye@rad{#1}}
  }
\newcommand{\eye}[1][]{% size, x, y, rotation  
\pgfkeys{/eye/.cd,
  x         = 0,
  y         = 0,
  rotation  = 0,
  radius    = 1
  } 
\pgfqkeys{/eye}{#1}   
   \draw[rotate around={\eye@rot:(\eye@x,\eye@y)}] 
         (\eye@x,\eye@y) -- ++(-.5*55:\eye@rad) 
         (\eye@x,\eye@y) -- ++(.5*55:\eye@rad);
   \draw (\eye@x,\eye@y) ++(\eye@rot+55:.75*\eye@rad) arc (\eye@rot+55:\eye@rot-55:.75*\eye@rad);
  % IRIS
   \draw[fill=gray] (\eye@x,\eye@y) ++(\eye@rot+55/3:.75*\eye@rad) arc (\eye@rot+180-55:\eye@rot+180+55:.28*\eye@rad);
  %PUPIL, a filled arc 
   \draw[fill=black] (\eye@x,\eye@y) ++(\eye@rot+55/3:.75*\eye@rad) arc (\eye@rot+55/3:\eye@rot-55/3:.75*\eye@rad);
}
\makeatother 


\makeatletter
\pgfkeys{/lineann/.cd,
  x/.code           = {\def\lineann@x{#1}},
  y/.code           = {\def\lineann@y{#1}},
  fill/.code    = {\def\lineann@fill{#1}},
  len/.code    = {\def\lineann@len{#1}},
  pos/.code    = {\def\lineann@pos{#1}},
  width/.code    = {\def\lineann@width{#1}},
  rotate/.code    = {\def\lineann@rotate{#1}},
  text/.code      = {\def\lineann@text{#1}}
  }
\newcommand{\lineann}[1][]{% size, x, y, rotation  
\pgfkeys{/lineann/.cd,
  x         = 0,
  y         = 0,
  len       =1,
  width       =1,
  fill     = white,
  pos     = 0.8,
  rotate     = 0,
  text      = {}
  } 
\pgfqkeys{/lineann}{#1}  
\begin{scope}[xshift=\lineann@x, yshift=\lineann@y]
    \begin{scope}[rotate=\lineann@rotate, black,inner sep=2pt, ]
        \draw[dashed, blue!40] (0,0) -- +(0,\lineann@len)
            node [coordinate, pos=\lineann@pos] (a) {};
        \draw[dashed, blue!40] (\lineann@width,0) -- +(0,\lineann@len)
            node [coordinate, pos=\lineann@pos] (b) {};
        \draw[|<->|] (a) -- node[fill=\lineann@fill, align=center, scale=0.8] {\lineann@text} (b);
    \end{scope}
    \end{scope} 
}
\makeatother 

\newcommand{\defcolor}[2]{
  \definecolor{SEviolet#1}{wave}{#2}
  \colorlet{#1}[rgb]{SEviolet#1}
}
\begin{document}
\begin{tikzpicture}[]
    \xdef\darkness{0}
    \xdef\opa{0.2}
    \xdef\SIZE{6}
    \xdef\setka{0}
    \input{setka}


    \eye[radius=0.5,x=2, y=8,rotation=-90] 


   
    \defcolor{cpp}{500}
    \draw[fill=cpp, opacity=0.8] (0,2) rectangle node [yshift=-0cm,opacity=1] {Фильтр} ++(4,0.5) ;

    \foreach \x in {1,1.1,...,3} {
        \pgfmathsetmacro{\infp}{125*\x+295}
      \defcolor{cpp\x}{\infp}
      \draw[color=cpp\x,->] (\x,0) -- (\x,2);
      % node[above] {\infp};
    }

    \foreach \x in {1.5,1.6,...,2.5} {
        \pgfmathsetmacro{\infp}{125*\x+295}
      \defcolor{cpp\x}{\infp}
      \draw[color=cpp\x,->] (\x,2.5) -- (\x,3.5);
    }

    \draw[fill=white, opacity=0.8] (0,3.5) rectangle node [yshift=0,opacity=1] {Поляризатор} ++(4,0.5) ;

    \foreach \x in {1.5,1.6,...,2.5} {
        \pgfmathsetmacro{\infp}{125*\x+295}
      \defcolor{cpp\x}{\infp}
      \draw[color=cpp\x,->] (\x,4) -- (\x,4.5);
    }

    \draw[fill=white, opacity=0.8] (0,4.5) rectangle node [ yshift=0cm,opacity=1] {Фазовая пластинка} ++(4,1);

    \lineann[x=0cm,y=4.5cm,len=1cm,width=1cm,text={$d$},pos=0.5,rotate=90]

    \foreach \x in {1.5,1.6,...,2.5} {
        \pgfmathsetmacro{\infp}{125*\x+295}
      \defcolor{cpp\x}{\infp}
      \draw[color=cpp\x,->] (\x,5.5) -- ++(0,0.5);
    }


    \draw[fill=white, opacity=0.8] (0,6) rectangle  node [ yshift=0cm,opacity=1] {Анализатор} ++(4,0.5);

    \foreach \x in {1.5,1.6,...,2.5} {
        \pgfmathsetmacro{\infp}{125*\x+295}
      \defcolor{cpp\x}{\infp}
      \draw[color=cpp\x,->] (\x,6.5) -- ++(0,1);
      % node[above] {\infp};
    }

\end{tikzpicture}    
\end{document}